\documentclass[a4j,12pt,]{jarticle}
 \usepackage{float}
 \usepackage{siunitx} %%SI単位系用
 \usepackage{amssymb, amsmath}
 \usepackage{ascmac,here,txfonts}
 \usepackage{hyperref}
 \usepackage{listings}
 \usepackage{pxjahyper}
 \usepackage[dvipdfmx]{graphicx}
 \usepackage{amssymb, amsmath}
  \usepackage{listings}
  \usepackage[dvipdfmx]{color}
 
 \lstset{
   language={Python},
   basicstyle={\ttfamily},
   identifierstyle={\small},
   commentstyle={\small\itshape},
   keywordstyle={\small\bfseries},
   ndkeywordstyle={\small},
   stringstyle={\small\ttfamily},
   frame={single},
   breaklines=true,
   columns=[l]{fullflexible},
   numbers=left,
   xrightmargin=0zw,
   xleftmargin=3zw,
   numberstyle={\scriptsize},
   stepnumber=1,
   numbersep=1zw,
   lineskip=-0.5ex,
 }
\begin{document}

{\noindent\small 第14回報告書 \hfill\today}
\begin{center}
  {\Large 自作関数により生成した日射量データの位相差分析}
\end{center}
\begin{flushright}
  祖父江匠真 \\
\end{flushright}

\section{概要}
今回は, 自作関数で生成した日射量データと, これを複製し, 時刻データにずれ時間を加えたデータとの間の位相差を計算して, ずれ時間と位相差の関係を調査した.

\section{自作関数で生成した日射量データについて}
位相差の計算には, 自作関数で生成した図 \ref{p1} に示す日射量データを使用した.

この日射量データは, 午前6時に立ち上がり, 正午に1 \si{\kilo\watt/m^2}を通り, 午後6時に0 \si{\kilo\watt/m^2}を通るようになっている.

\begin{figure}[H]
  \begin{center}
    \includegraphics[width=160mm]{simulated.png}
    \caption{自作関数で生成した日射量データ}
    \label{p1}
  \end{center}
\end{figure}

\section{位相差を求める手順}

ずれ時間とその値に対応する位相差の計算は, 以下手順を繰り返し実行して行った.

\begin{enumerate}
  \item scipy.fft.fft メソッドを使用して, 自作関数で生成した日射量データと複製データの高速フーリエ変換(FFT)を行い, 振幅スペクトルを取得する.  
  \item numpy.abs メソッドを使用して, 振幅スペクトルで最も振幅が大きい周波数に対応するフーリエ係数の位相をnumpy.angleメソッドを用いて求め, 自作関数で生成した日射量データと複製データの位相差を計算する. 
  \item numpy.roll 関数を使用して複製データの時刻データを1sずつ遅らせる方向にずらす. 
\end{enumerate}

本検証では, この計算を0sから1000sまでのずれ時間で繰り返し行った. 

\section{結果}
結果として, 図 \ref{p2}に示すグラフが得られた. 横軸がずらした時間の秒数, 縦軸が位相差となっている. 図 \ref{p2}から, 自作関数で生成した日射量データを用いた場合, ずれ時間を加えていない複製データとの間の位相差は0 \si{\radian}となった.

\begin{figure}[H]
  \begin{center}
    \includegraphics[width=160mm]{phase_difference.png}
    \caption{ずれ時間と位相差}
    \label{p2}
  \end{center}
\end{figure}

\section{まとめ}
今回は, 自作関数で生成した日射量データと, これを複製し, 時刻データにずれ時間を加えたデータとの間の位相差を計算して, ずれ時間と位相差の関係を調査した.

結果として, ずれ時間に対して位相差が線形的に変化し, 自作関数で生成した日射量データと, ずれ時間を加えていない複製データの位相差は0 \si{\radian}となった.

次回は, 実測データとPblivで生成したシミュレーションデータについて, 複数の日付でシミュレーションデータにずれ時間を加えていない条件における実測データとの位相差を計算し, 日付が変わることによって位相差がどのように変化するかを調査する.
\end{document}